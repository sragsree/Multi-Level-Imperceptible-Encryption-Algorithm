\documentclass[a4paper, 12pt, notitlepage]{report}

\usepackage{amsfonts} % if you want blackboard bold symbols e.g. for real numbers
\usepackage{graphicx} % if you want to include jpeg or pdf pictures
\usepackage{geometry}
\usepackage{subfigure}
\usepackage[latin1]{inputenc}
\usepackage{tikz}
\usepackage[demo]{graphicx}
\usepackage{babel,blindtext}
\usetikzlibrary{shapes.geometric, arrows}

\geometry{verbose,a4paper,tmargin=30mm,bmargin=30mm,lmargin=30mm,rmargin=30mm}


\begin{document}

%%%%%%%%%% PRELIMINARY MATERIAL %%%%%%%%%%
\begin{titlepage}
\newcommand{\horrule}[1]{\rule{\linewidth}{#1}} 	% Horizontal rule
\begin{center}
		\usefont{OT1}{bch}{b}{n}
		\horrule{0.5pt} \\
		\usefont{OT1}{bch}{b}{n}
		\huge INFORMATION HIDING IN AUDIO SIGNALS \\
		\horrule{2pt} \\
		\end{center}
		{\centering
		\vspace{1cm}
		\\project/btech/ece@nssce/1014/16.22\\
		\vspace{1cm}
\large\bf{\emph{PROJECT REPORT}}\\

\it{As a partial fulfilment of the curriculum}\\
\vspace{0.25cm}
\vspace{0.1cm}
\vspace{1cm}
\it
by \\
\vspace{0.5cm}
\rm
{\large \bf {\emph{RAHUL R NAIR (NSAKEEC066)\\ SREERAG S (NSAKEEC086)\\VISHNU K (NSAKEEC097)\\VISHNU S DEV (NSAKEEC099)}}}\\

\vspace{1.0cm}

{\it{under the guidance of}} \\
\vspace{0.5cm}
\hspace{0.05cm} {\large \bf {Dr.Sindhu.R}}\\
\hspace{0.025cm} {\large \bf {Head Of The Department}}\\
\vspace {0.5cm}
\vspace {0.5cm}
\begin{figure}[h]
{\centering {\includegraphics[scale=0.8]{logo.png}}\par}
\end{figure}
\begin{center}
Department of Electronics \& Communication Engineering \\
NSS College OF Engineering, Palakkad \\March 2014
\end{center}
}
\end{titlepage}

\newpage
\section*{
\begin{center}
NSS College OF Engineering, Palakkad  \\
\end{center}}

\begin{figure}[h!]
{\centering {\includegraphics[scale=0.85]{logo.png}}\par}
\end{figure}

\begin{center}
\textbf{Department of Electronics \& Communication Engineering}  \\
\end{center}}

\section*{
\begin{center}
CERTIFICATE \\
\end{center}}\\

\subsubsection{\vspace{0.25cm}Certified that project work titled INFORMATION HIDING IN AUDIO SIGNALS is a bonafied work carried out in eighth semester by Rahul R\vspace{0.25cm} Nair, Sreerag S, Vishnu K and Vishnu S Dev during the academic year\vspace{0.25cm} of 2013-2014 under the guidance Dr.Sindhu.R,Head of the department\vspace{0.25cm} as part of the partial fulfillment for the award of Bachelor of Technology in\vspace{0.25cm} ELECTRONICS AND COMMUNICATION ENGINEERING from\vspace{0.25cm}    University of Calicut and no part of this work has been submitted earlier for\vspace{0.25cm} the award of any degree.}\\
\vspace{3.0cm}
\textbf{PROJECT GUIDE} \hspace{5.0cm} \textbf{HEAD OF DEPARTMENT}\\
\vspace{0.25cm}
Dr.Sindhu R \hspace{6.8cm} Dr. Sindhu . R
\newpage
\section*{\begin{center}{ACKNOWLEDGEMENT}\end{center}} % this must be included in undergradate projects
\vspace{1.0in}
We express our sincere piece of gratitude to Dr. Sindhu . R ,Head Of Electronics And Communication Department for her moral support and encouragement as Project Guide\\

We would like to express our deep sense of gratitude to all the faculity members, Department of Electronics and Communication, for being a great force behind all our efforts through their advice, guidance, and encouraging nature through out the project duration. \\

We acknowledge the help of Prof. Sidharth.N and for his guidance and motivation throughout the project.\\
 
Finally we would like express our gratitude towards our parents for their moral and financial support for making this project a success\\

\vspace{5.0cm}
\begin{flushright}
{RAHUL R NAIR (NSAKEEC066)\\ SREERAG S (NSAKEEC086)\\VISHNU K (NSAKEEC097)\\VISHNU S DEV (NSAKEEC099)\\
\end{flushright}
\vspace{1.0cm}
March 2014\\
Department of Electronics \& Communication Engineering\\
NSS College Of Engineering,Palakkad
\newpage

\newpage
\section*{\begin{center}{ABSTRACT}\end{center}}
\\

Data  hiding,  a  form  of  steganography,  is  one  of  the  emerging  techniques  that embeds  secret  data  into  a  digital media  and  thus  ensures  secured  data  transfer.  In  this project,  the  steganographic  method  used,  is  based  on  audio  steganography  which  is concerned  with  embedding  secret  data  in  an  audio  file.\\

Data hiding in media, including images, video, and audio, as well as in data files is currently of great interest both commercially, mainly for the protection of copyrighted digital media, and to the government and law enforcement in the context of information systems security and covert communications.\\

Embedding secret messages using audio signal in digital format is now  the  area  in  focus.  There  exist  numerous  steganography techniques for hiding information in audio medium.\\

In  this project,  a  new  model  ISS-IHAS  -  embedding message in audio signal that embeds the message like the existing systems but  with  strong  encryption  that  gains  the  full  advantages  of cryptography.  Using  this technique  it  is  possible  to  conceal  the full  existence  of  the  original  message  and  the  results  obtained  from the  proposed  model  is  compared  with  other  existing on the basis of Signal To Noise ratio and MER.\\
  
The techniques that are in consideration are LSB, Parity, Phase, Echo and Spread spectrum Encoding schemes. Each of them are attempted to be implemented and demonstatred with the aid of MATLAB.

\tableofcontents
\listoffigures 

%%%%%%%%%% MAIN TEXT STARTS HERE %%%%%%%%%%

%%%%%%%%%% SAMPLE CHAPTER %%%%%%%%%%
\chapter{INTRODUCTION}

A number of different cover objects (signals) can be used to carry hidden messages. Data hiding in audio signals exploits imperfection of human auditory system known as audio masking. In presence of a loud signal (masker), another weaker signal may be inaudible, depending on spectral and temporal characteristics of both masked signal and masker Masking models are extensively studied for perceptual compression of audio signals. In the case of perceptual compression the quantization noise is hidden below the masking threshold, while in a data hiding application the embedded signal is hidden there. Data hiding in audio signals is especially challenging, because the human auditory system operates over a wide dynamic range. \\

\begin{figure}[h!]
{\centering {\includegraphics[scale=0.85]{intro_block.jpg}}\par}
\end{figure}\\

The human auditory system perceives over a range of power greater than one billion to one and a range of frequencies greater than one thousand to one. Sensitivity to additive random noise is also acute. The perturbations in a sound file can be detected as low as one part in ten million (80 dB below ambient level). However, there are some “holes” available. While the human auditory system has a large dynamic range, it has a fairly small differential range. As a result, loud sounds tend to mask out quiet sounds. Additionally, the human auditory system is unable to perceive absolute phase, only relative phase. Finally, there are some environmental distortions so common as to be ignored by the listener in most cases.\\
 
Many number of algortihms are available for the above described purpose and out of them most used ones are taken here for implementation and analysis.

\chapter{LSB CODING}
Least  significant  bit  (LSB)  coding  is  the  simplest  way  to embed  information  in  a  digital  audio  file.  By  substituting  the least  significant  bit  of  each  sampling  point  with  a  binary message,  LSB  coding  allows  for  a  large  amount  of  data  to  be encoded.One of the earliest techniques  studied  in  the  information  hiding  of  digital  audio  (as 
well  as  other  media types)  is  LSB  coding.  In  this  technique  LSB of  binary  sequence  of  each  sample  of  digitized  audio  file  is replaced with binary equivalent of secret message. For  example  if we want to hide the letter 'A' (binary equivalent 1000001)  into  a   digitized  audio  file  where  each  sample  is represented with 16 bits, then LSB of 7 consecutive samples (each of 16 bit size) is replaced with each bit of binary equivalent of the letter'A' .\\ 

It  is  the  simplest  way  to  embed  information  in  a digital  audio  file.  It  allows  large  amount  of  data  to  be  concealed within an audio file, use of only one LSB of the host audio sample gives a capacity equivalent to the sampling rate which could vary from 8  kbps to 44.1 kbps (all samples used). This method is more  widely  used  as  modifications  to  LSBs  usually  not  create audible  changes to  the  sounds.It  has  considerably low robustness against attacks.\\   

In  LSB  coding,  the  ideal  data  transmission  rate  is  1  kbps per 1 kHz. In some implementations of LSB coding, however, 
the two least significant bits of a sample are replaced with two message  bits.  This  increases  the  amount  of  data  that  can  be 
encoded but also increases the amount of resulting noise in the audio file as well. Thus, one should consider the signal content 
before  deciding  on  the  LSB  operation  to  use.  For  example,  a sound file that was recorded in a bustling subway station would 
mask  low-bit  encoding  noise.  On  the  other  hand,  the  same noise would be audible in a sound file containing a piano solo.\\ 

To  extract  a  secret  message  from  an  LSB  encoded  sound file, the receiver needs access to the sequence of sample indices 
used  in  the  embedding  process.  Normally,  the  length  of  the secret  message  to be  encoded  is  smaller  than the  total  number of  samples  in  a  sound  file.  One  must  decide  then  on  how  to choose  the  subset  of  samples  that  will  contain  the  secret message  and  communicate  that  decision  to  the  receiver.  One trivial  technique  is  to  start  at  the  beginning  of  the  sound  file and  perform  LSB  coding  until  the  message  has  been completely  embedded,  leaving  the  remaining  samples 
unchanged. This creates a security problem, however in that the first  part  of  the  sound  file  will  have  different  statistical 
properties  than  the  second  part  of  the  sound  file  that  was  not modified.  One  solution  to  this  problem  is  to  pad  the  secret message with random bits so that the length of the message is equal  to the  total  number  of samples.  Yet  now  the  embedding process  ends  up  changing  far  more  samples  than  the transmission  of  the  secret  required.  This  increases  the 
probability  that  a  would-be  attacker  will  suspect  secret communication.\\

\begin{block}{\textbf{Steps for Data Embedding}}
\begin{enumerate}
  \item {
    Read the cover audio signal.  
  }
	\item {
    Read the audio signal to be embedded, its size less than the size of the cover audio signal and convert it into binary sequence of bits.
  }
	\item {
    Every message bit is embedded into the LSBs of the cover audio after processing. 
  }
	\item {
    Processing is done as follows:
		
		\begin{itemize}
			\item If the message bit to be embedded is 0, then adjust or flip the LSB such that the XORing of LSB and next to LSB is 0.
			\item If the message bit to be embedded is 1, then adjust or flip the LSB such that the XORing of LSB and next to LSB is 1.
		\end{itemize}
	\item The modified cover audio samples are then written to the file forming the stego audio signal. 	
  }
  \end{enumerate}
\end{block}\\

\begin{block}{\textbf{Steps for Data Extraction/Retrieval:}}
\begin{enumerate}
  \item {
    The Stego audio file is read.  
  }
	\item {
   Retrieval of the  message  bit  is  done  by  XORing  the LSB and the bit next to LSB. If the result of XORing is 0, then the message bit is 0. If the result of XORing is 1, then the message bit is 1. 
  }
	\item {
    After every such 16 message bits are retrieved, they are converted to their decimal equivalents. 
  }
	\item {
    Finally the secret signal is reconstructed. }
  \end{enumerate}
\end{block}



\section{Encoder Implementation}
\\

Being the simplest algorithm of all,the code is developed without any complexity.Initially the audio wave file is read,they are sampled to data samples based on the lenght of the audio signal,then they are converted to equivalent binary form,the resulting matrix is converted to decimal.The MSB of the sample is used to store the sign.To start the stegnography procedure we accept the message signal, it is then transformed into a row vector. The length of the message is emedded in the first 16 samples, embedding of secret message starts from the 17th sample onwards.To regenerate the audio signal it is then converted to its analog equivalent.\\

% Define block styles
\tikzstyle{decision} = [diamond, draw, fill=blue!20, 
    text width=4.5em, text badly centered, node distance=3cm, inner sep=0pt]
\tikzstyle{block} = [rectangle, draw,
    text width=35em, text centered, rounded corners, minimum height=2em]
\tikzstyle{line} = [draw, -latex']
\tikzstyle{cloud} = [draw, ellipse,fill=red!20, node distance=3cm,
    minimum height=2em]

		
\begin{tikzpicture}[node distance = 1.5cm, auto]
		\node [block] (init) {Initialisation};
    \node [block, below of=init] (identify) {Read Wave file};
    \node [block, below of=identify] (evaluate) {Analog to Digital Conversion};
		\node [block, below of=evaluate] (evaluate1) {Change Samples to decimal};
		\node [block, below of=evaluate1] (evaluate2) {Use MSB to Store Sign };
		\node [block, below of=evaluate2] (evaluate3) {Start Stegnography Procedure};
		\node [block, below of=evaluate3] (evaluate4) {Accept Secret Message};
		\node [block, below of=evaluate4] (evaluate5) {Form Row Vector of Secret Message};
		\node [block, below of=evaluate5] (evaluate6) {Embed Message Length in First 16 Samples};
		\node [block, below of=evaluate6] (evaluate7) {Embed Secret Message from 17th Sample};
		\node [block, below of=evaluate7] (evaluate8) {Digital To analog Conversion};
		\node [block, below of=evaluate8] (evaluate9) {Check The Sign From MSB};
		\node [block, below of=evaluate9] (evaluate10) {Save The Sound With Secret Message};
    \node [block, below of=evaluate10] (stop) {Stop};
		
		\path [line] (init) -- (identify);
    \path [line] (identify) -- (evaluate);
		\path [line] (evaluate) -- (evaluate1);
		\path [line] (evaluate1) -- (evaluate2);
		\path [line] (evaluate2) -- (evaluate3);
		\path [line] (evaluate3) -- (evaluate4);
		\path [line] (evaluate4) -- (evaluate5);
		\path [line] (evaluate5) -- (evaluate6);
		\path [line] (evaluate6) -- (evaluate7);
		\path [line] (evaluate7) -- (evaluate8);
		\path [line] (evaluate8) -- (evaluate9);
		\path [line] (evaluate9) -- (evaluate10);
    \path [line] (evaluate10) -- (stop);
    
\end{tikzpicture}\\

\section{Decoder Implementation}
\\
The decoder in effect performs just the reverse of what the encoder does. It first reads the audio signal with secret message change the samples to binary equivalents.Now the MSB of each sample is checked to determine the sign.The scanning of first 16 wave data samples gives an idea about the length of the message. After determining the length the upper limit for scanning of message is determined and the extraction procedure starts,a row vector of extracted bits is formed and is converted to a nX8 matrix.The ASCII equivalents of these values is compared and the secret message is displayed in command window. \\     


\tikzstyle{decision} = [diamond, draw, fill=blue!20, 
    text width=4.5em, text badly centered, node distance=3cm, inner sep=0pt]
\tikzstyle{block} = [rectangle, draw,
    text width=35em, text centered, rounded corners, minimum height=2em]
\tikzstyle{line} = [draw, -latex']
\tikzstyle{cloud} = [draw, ellipse,fill=red!20, node distance=3cm,
    minimum height=2em]

		
\begin{tikzpicture}[node distance = 1.5cm, auto]
		\node [block] (init) {Initialisation};
   
    \node [block, below of=init] (identify) {Read Wave file};
    \node [block, below of=identify] (evaluate) {Analog to Digital Conversion};
		\node [block, below of=evaluate] (evaluate1) {Change Samples to decimal};
		\node [block, below of=evaluate1] (evaluate2) {Check MSB to get the Sign };
		\node [block, below of=evaluate2] (evaluate3) {Check First 16 Samples to get Message Length};
		\node [block, below of=evaluate3] (evaluate4) {Begin Extraction of Secret Message};
		\node [block, below of=evaluate4] (evaluate5) {Form Row Vector of Secret Message};
		\node [block, below of=evaluate5] (evaluate6) {Change Row Vector to nX8 Matrix};
		\node [block, below of=evaluate6] (evaluate7) {Check for ASCII values};
		\node [block, below of=evaluate7] (evaluate8) {Reconstruct the Message from ASCII};
		\node [block, below of=evaluate8] (evaluate9) {Display Secret Message};
    \node [block, below of=evaluate9] (stop) {Stop};
		
		\path [line] (init) -- (identify);
    \path [line] (identify) -- (evaluate);
		\path [line] (evaluate) -- (evaluate1);
		\path [line] (evaluate1) -- (evaluate2);
		\path [line] (evaluate2) -- (evaluate3);
		\path [line] (evaluate3) -- (evaluate4);
		\path [line] (evaluate4) -- (evaluate5);
		\path [line] (evaluate5) -- (evaluate6);
		\path [line] (evaluate6) -- (evaluate7);
		\path [line] (evaluate7) -- (evaluate8);
		\path [line] (evaluate8) -- (evaluate9);
    \path [line] (evaluate9) -- (stop);
    
\end{tikzpicture}\\

\begin{figure}[h!]
{\centering {\includegraphics[scale=0.47]{Sample_LSB.jpg}}\par}
\caption{Diagramatic representation of procedure for using XOR with LSB's}
\end{figure}\\

\section{Simulation Result}

The following is a screenshot of the Simulated result of the LSB coding technique.It shows the plot of an audio signal before and after the encoding procedure 

\begin{figure}[h!]
{\centering {\includegraphics[scale=0.8]{LSB_resuilt.jpg}}\par}
\caption{LSB Coding result}
\end{figure}\\
\newpage
\section{Performance}

The performance of the code is evaluated in terms of SNR and MER with varying cover bits and embedded bits.\\


\begin{table}[h]
\begin{tabular}{|l|l|l|l|l|}
\hline
\textbf{ Audio(.wav)}&\textbf{Cover Bits(KB)}&\textbf{Msg length}&\textbf{MER}&\textbf{SNR(db)} \\ \hline
 Button&225  &24  &2.4233  &46.93  \\ \hline
 Beep&178.1  &24  &2.4756  &48.93  \\ \hline
 One&79  &24  &2.5642  &50.649  \\ \hline
\end{tabular}
\caption{By varying cover bits}
\end{table}

Message length was kept constant and the cover bits count was kept on varying, and it was found that even for a large increase in cover bits the MER varies only in the range of 2.4 to 2.5, which implies that the code is not effected by errors,however the SNR varies significantly.When the cover bits were 178.1KB the SNR was 48.1792db and for 79KB SNR was 50.649db, which means length of cover bits is in inverse propotion with SNR.

\begin{figure}[h!]
{\centering {\includegraphics[scale=0.45]{LSB_Comp1_MER_graph.jpg}}\par}
\caption{MER plot}
\end{figure}\\

The Inference from MER plot is that MER changes in propotion with cover bit count but there is a only a very less variation for MER even for large increase in cover bit count.

\begin{figure}[h!]
{\centering {\includegraphics[scale=0.45]{LSB_Comp1_SNR_graph.jpg}}\par}
\caption{SNR plot}
\end{figure}\\

The SNR plot shows that there is a reasonably large change in SNR for change in cover bits, while comparing the values with the constant cover bit table its quite evident that SNR is in inverse propotion with length of cover bits. 

\begin{table}[h]
\begin{tabular}{|l|l|l|l|l|}
\hline
\textbf{ Audio(.wav)}&\textbf{Cover Bits(KB)}&\textbf{Msg length}&\textbf{MER}&\textbf{SNR(db)} \\ \hline
 Beep&178.1  &24  &2.4233  &48.1792  \\ \hline
 Beep&178.1  &168  &2.7747  &50.2414  \\ \hline
 Beep&178.1  &1000  &2.1542  &55.45  \\ \hline
\end{tabular}
\caption{Using Beep.wav}
\end{table}

In this particular analysis cover bits is varied keeping the embedded bits as constant.When the message length was 1000 the SNR was found to be 55.45db, also the MER was found as 3.1542, where as when message length reduced to 24 the SNR also reduced to 48.1792db and MER reduced to 2.4233 which implies SNR and MER varies in propotion with message length.  

\begin{table}[h]
\begin{tabular}{|l|l|l|l|l|}
\hline
\textbf{ Audio(.wav)}&\textbf{Cover Bits(KB)}&\textbf{Msg length}&\textbf{MER}&\textbf{SNR(db)} \\ \hline
 Button&225  &24  &2.4756  &46.93  \\ \hline
 Button&225  &168  &2.5236  &47.93  \\ \hline
 Button&225  &1000  &1.1544  &55.4107  \\ \hline
\end{tabular}
\caption{Using Button.wav}
\end{table}

The cover bits are kept constant but having higher value,the emededded bit count is chosen to be 24,168,1000.When the message length was 24 MER was recorded as 2.4756 and SNR was 46.93 but as we incresed the count to 1000, MER was recorded as 2.5642  and SNR as 51.4107 which implies SNR and MER varies in propotion with message length.

\begin{table}[h]
\begin{tabular}{|l|l|l|l|l|}
\hline
\textbf{ Audio(.wav)}&\textbf{Cover Bits(KB)}&\textbf{Msg length}&\textbf{MER}&\textbf{SNR(db)} \\ \hline
 One&79  &24  &2.5642  &50.649  \\ \hline
 One&79  &168  &2.8024  &52.7337  \\ \hline
 One&79  &1000  &3.0582  &54.3946  \\ \hline
\end{tabular}
\caption{Using One.wav}
\end{table}

As the sampling rate was reduced the SNR was decreasing propotionally,LSB being a very popular and common technique used for stegnography it was analysed it different situations.The embedded bits were taken to be constant and and cover bits were varied,Mean error rate changed propotionally to cover bits which implies that the the code with hidden message gets destroyed very easily if only a small message is hidden in a relatively large number of cover bits as this technique proposes the hiding of message bits in the least significant bit,if any of the LSB of any sample is effected the whole message gets corrupted.The SNR performance of the code is relatively good,as the sampling rate is increased we can expect the best out of this code.As there is no complexity in algorithm, in security point of view the code is advised to consider for confidential data hiding.More over this coding is at its best when the hiding rate is limited to 16kbps.\\

\chapter{PARITY CODING}
One of the prior works in audio data hiding technique is parity coding technique. Instead of breaking a signal down into individual samples, the parity coding method breaks a signal down into separate regions of samples and encodes each bit from the secret message in a sample region's parity bit. If the parity bit of a selected region does not match the secret bit to be encoded, the process flips the LSB of one of  the samples in the region. Thus, the sender has more of a choice in encoding the secret bit, and the signal can be changed in a more unobtrusive fashion. Figure 3.1, shows the parity coding procedure.\\

\begin{figure}[h!]
{\centering {\includegraphics[scale=0.75]{Methodologies_parity.jpg}}\par}
\caption{Parity Coding Procedure}
\end{figure}\\

Instead of breaking  a  signal  down into individual  samples,the  parity  coding  method  breaks  a  signal  down  into  separate regions  of  samples  and  encodes  each  bit  from  the  secret message  in  a  sample  region's  parity  bit.  If  the  parity  bit  of  a selected region does not match the secret bit to be encoded, the process flips the LSB of one of the samples in the region. Thus, the sender has more of a choice in encoding the secret bit, and the signal can be changed in a more unobtrusive fashion. 
Using  the  parity  coding  method,  the  first  three  bits  of  the message 'HEY' are encoded in the following figure. Even parity 
is desired. The decoding process extracts the secret message by calculating  and  lining up  the  parity  bits  of  the  regions  used  in the encoding process. Once again, the sender and receiver can use  a  shared  secret  key  as  a  seed  in  a  pseudorandom  number generator to produce the same set of sample regions.\\

\begin{block}{\textbf{Steps for Data embedding:}}
\begin{enumerate}
  \item {
    Read the cover audio signal.  
  }
	\item {
   Read the audio signal to be embedded, its size less than the size of the cover audio signal and convert it into binary sequence of message bits.
  }
	\item {
Depending upon the value of the message bit to be embedded (0/1), the LSB of the sample of the cover audio signal is modified or unchanged. 
  }
	\item {
If the message bit to be embedded is 0, then the LSB of the sample of the cover audio signal is modified or unchanged such that the parity of the sample after embedding of this message bit is even.  
}
\item If the message bit to be embedded is 1, then the LSB of the sample of the cover audio signal is modified or unchanged   
    such that the parity of the sample after embedding of this message bit is odd.
\item The modified cover audio samples are then written to the file forming the stego audio signal
  \end{enumerate}
\end{block}\\

\begin{block}{\textbf{Steps for Data Extraction/Retrieval:}}
\begin{enumerate}
  \item {
    The Stego audio file is read.  
  }
	\item {
The parity of every sample of the stego is checked. 
  }
	\item {
    If the parity is even, then the message bit retrieved is 0.
  }
	\item {
    If the parity is odd, then the message bit retrieved is 1.}
	\item After every such 16 message bits are retrieved, they are converted to decimal equivalents.
	\item Finally the secret message is reconstructed.
  \end{enumerate}
\end{block}\\

 The  sender  has  more  of  a  choice  in  encoding  the secret  bit,  and  the  signal  can  be  changed  in  a  more  unobtrusive 
manner.This method like LSB coding is not robust in nature.\\

\section{Encoder Implementation}
\\
The parity coder is actually an improved version of LSB coder.The parity code is developed with a GUI for interactions.When the code is excecutes the MATLAB M-fig file gets initialised in the background before we actually see the GUI.To encode a message audio file of wave format is read and the message to embed is taken as input via the GUI.Before the actual encoding starts the code determines the length of the message and length of the cover audio and finds out whether the message can be embedded in the intended cover signal.If it is compatible the message is converted to binary sequence and is broken down to many samples.The embedding procedure then starts and data is hided by adjusting the parity of the samples.Concatenate the samples to obtain audio signal back.\\ 

\tikzstyle{decision} = [diamond, draw, fill=blue!20, 
    text width=4.5em, text badly centered, node distance=3cm, inner sep=0pt]
\tikzstyle{block} = [rectangle, draw,
    text width=35em, text centered, rounded corners, minimum height=2em]
\tikzstyle{line} = [draw, -latex']
\tikzstyle{cloud} = [draw, ellipse,fill=red!20, node distance=3cm,
    minimum height=2em]

		
\begin{tikzpicture}[node distance = 1.5cm, auto]
		\node [block] (init) {Initialisation};
    \node [block, below of=init] (identify) {Read the cover file};
    \node [block, below of=identify] (evaluate) {Accept the message to be embedded};
		\node [block, below of=evaluate] (evaluate1) {Check that message size less than cover size};
		\node [block, below of=evaluate1] (evaluate2) {Convert message to binary sequence};
		\node [block, below of=evaluate2] (evaluate3) {Sample cover message};
		\node [block, below of=evaluate3] (evaluate4) {Checking message to be embedded and adjusting parity of samples};
		\node [block, below of=evaluate4] (evaluate5) {Concatenate samples to get back Stego file};
    \node [block, below of=evaluate5] (stop) {stop};
		
		\path [line] (init) -- (identify);
    \path [line] (identify) -- (evaluate);
		\path [line] (evaluate) -- (evaluate1);
		\path [line] (evaluate1) -- (evaluate2);
		\path [line] (evaluate2) -- (evaluate3);
		\path [line] (evaluate3) -- (evaluate4);
		\path [line] (evaluate4) -- (evaluate5);
    \path [line] (evaluate5) -- (stop);
    
\end{tikzpicture}\\

\section{Decoder Implementation}
\\
As in the encoder, the GUI is initialised first.Then the cover signal with hidden message is read, to get the parity value of each sample, cover signal is analysed after sampling it, an empty matrix is initialised and message value in accordance with the parity value is recorded in that matrix,convert these integer values to binary values and then binary to decimal equivalents, the ASCII values corresponding each decimal is recorded and the message is displayed.\\  

\tikzstyle{decision} = [diamond, draw, fill=blue!20, 
    text width=4.5em, text badly centered, node distance=3cm, inner sep=0pt]
\tikzstyle{block} = [rectangle, draw,
    text width=35em, text centered, rounded corners, minimum height=2em]
\tikzstyle{line} = [draw, -latex']
\tikzstyle{cloud} = [draw, ellipse,fill=red!20, node distance=3cm,
    minimum height=2em]

		
\begin{tikzpicture}[node distance = 1.5cm, auto]
		\node [block] (init) {Initialisation};
    \node [block, below of=init] (identify) {Read the Stego file};
    \node [block, below of=identify] (evaluate) {Sample the stego file};
		\node [block, below of=evaluate] (evaluate1) {Consider each sample and xor each bit of sample and get parity value};
		\node [block, below of=evaluate1] (evaluate2) {Create an empty matrix};
		\node [block, below of=evaluate2] (evaluate3) {Add to it message values in accordance to parity values};
		\node [block, below of=evaluate3] (evaluate4) {Convert to int value from bin value};
		\node [block, below of=evaluate4] (evaluate5) {Convert binary to decimal values} ;
		\node [block, below of=evaluate5] (evaluate6) {Reconstruct and display the message};
    \node [block, below of=evaluate6] (stop) {stop};

		
		\path [line] (init) -- (identify);
    \path [line] (identify) -- (evaluate);
		\path [line] (evaluate) -- (evaluate1);
		\path [line] (evaluate1) -- (evaluate2);
		\path [line] (evaluate2) -- (evaluate3);
		\path [line] (evaluate3) -- (evaluate4);
		\path [line] (evaluate4) -- (evaluate5);
		\path [line] (evaluate5) -- (evaluate6);
    \path [line] (evaluate6) -- (stop);
    
\end{tikzpicture}\\

\section{Simulation Result}

The following is a screenshot of the simulated result of the parity coding technique.It shows the plot of an audio signal before and after the encoding procedure 
\newpage

\begin{figure}[h!]
{\centering {\includegraphics[scale=0.45]{parity_comp1_new2.jpg}}\par}
\caption{Parity Coding result}
\end{figure}\\		
		
\section{Performance}

The performance of the code is evaluated in terms of SNR and MER with varying cover bits and embedded bits.

\begin{table}[h]
\begin{tabular}{|l|l|l|l|l|}
\hline
\textbf{ Audio(.wav)}&\textbf{Cover Bits(KB)}&\textbf{Msg length}&\textbf{MER}&\textbf{SNR(db)} \\ \hline
 Button&225  &24  &3.7229  &54.9655  \\ \hline
 Beep&178.1  &24  &2.9294  &52.1792  \\ \hline
 One&79  &24  &2.837  &51.2007  \\ \hline
\end{tabular}
\caption{By varying cover bits}
\end{table}

Message length was kept constant and the cover bits count was kept on varying, and it was found that for a large increase in size of cover bits the MER varies only in the range of 3.7 to 2.8, which implies that the code is very much effected by errors,which makes it not suitable for long diatance transmission, SNR also varies significantly for increase in cover bit count.When the cover bits were 178.1KB the SNR was 51.328db and for 79KB SNR was 51.2007db, which means length of cover bits is in direct propotion with SNR.\\

\newpage
\begin{figure}[h!]
{\centering {\includegraphics[scale=0.45]{parity_comp1_MER_graph.jpg}}\par}
\caption{MER plot}
\end{figure}\\

The Inference from MER plot is that MER changes in propotion with cover bit count but there is a only a slight variation for MER even for large increase in cover bit count and it is in direct propotion with cover bit count.The variation is also very large even for a slight change in cover bits.\\ 

\begin{figure}[h!]
{\centering {\includegraphics[scale=0.45]{parity_Comp1_SNR_graph.jpg}}\par}
\caption{SNR plot}
\end{figure}\\

The SNR plot shows that there is a reasonably large change in SNR for change in cover bits,while comparing the values with the constant cover bit table its quite evident that SNR is in direct propotion with length of cover bits.\\ 

\begin{table}[h]
\begin{tabular}{|l|l|l|l|l|}
\hline
\textbf{ Audio(.wav)}&\textbf{Cover Bits(KB)}&\textbf{Msg length}&\textbf{MER}&\textbf{SNR(db)} \\ \hline
 Beep&178.1  &24  &2.9294  &51.2416  \\ \hline
 Beep&178.1  &168  &3.1361  &52.7834  \\ \hline
 Beep&178.1  &1000  &3.4071  &53  \\ \hline
\end{tabular}
\caption{Using Beep.wav}
\end{table}

In this particular analysis cover bits is varied keeping the embedded bits as constant.When the message length was 1000 the SNR was found to be 53db,also the MER was found as 3.4071,whereas when message length reduced to 24 the SNR also reduced to 51.2416 and MER reduced to 2.9294 which implies SNR and MER varies in propotion with message length, in comparison with LSB the MER and SNR values has increased.\\

\begin{table}[h]
\begin{tabular}{|l|l|l|l|l|}
\hline
\textbf{ Audio(.wav)}&\textbf{Cover Bits(KB)}&\textbf{Msg length}&\textbf{MER}&\textbf{SNR(db)} \\ \hline
 Button&225  &24  &3.7229  &54.9655  \\ \hline
 Button&225  &168  &3.8211  &55.1002  \\ \hline
 Button&225  &1000  &4  &55.4454  \\ \hline
\end{tabular}
\caption{Using Button.wav}
\end{table}

The cover bits are kept constant but having higher value, the emededded bit count is chosen to be 24,168,1000.When the message length was 24, MER was recorded as 3.7229 and SNR was 54.9655 but as we incresed the count to 1000, MER was recorded as 4 and SNR as 55.4454 which implies SNR and MER varies in propotion with message length,in comparison with LSB the MER and SNR values has increased.\\

\begin{table}[h]
\begin{tabular}{|l|l|l|l|l|}
\hline
\textbf{ Audio(.wav)}&\textbf{Cover Bits(KB)}&\textbf{Msg length}&\textbf{MER}&\textbf{SNR(db)} \\ \hline
 One&79  &24  &2.837  &51.2007  \\ \hline
 One&79  &168  &3.0401  &51.7988  \\ \hline
 One&79  &1000  &3.3766  &52.6132  \\ \hline
\end{tabular}
\caption{Using One.wav}
\end{table}

Parity Coding is nothing but an improed version of LSB,rather than hiding the message in a LSB of a sample, the sample as a whole is used for hiding th message bit.From the first table its quite evident that as the size of the cover bits increases there is a noticable increase in SNR value.This implies that as the number of samples increases the performance of the code also increases.The MER here shows a realatively good range of values than the LSB which reveals that the chnces of error getting into the hdden message is less.In security point of view its least secure of all the algorithms mentioned in this comparison on the grounds that if we know that there is message embedded in the signal it can be decoded very easily using any general logic, if the sample size is known the work is even more easy.Here while doing the comparison it was noted that as the sample size is reduced,the number of samples increses which eventually leads to an increase in SNR.\\

\chapter{PHASE CODING}

The phase coding method works by substituting the phase of an initial audio segment with a reference phase that represents the data. The phase of subsequent segments is adjusted in order to preserve the relative phase between segments. Phase coding, when it can be used, is one of the most effective coding methods in terms of the signal-to perceived noise ratio.When the phase relation between each frequency component is dramatically changed, noticeable phase dispersion will occur. However, as long as the modification of the phase is sufficiently small (sufficiently small depends on the observer; professionals in broadcast radio can detect modifications that are imperceivable to an average observer), an inaudible coding can be achieved. Phase coding relies on the fact that the phase components of sound are not as perceptible to the human ear as noise is. Rather than introducing perturbations, the technique encodes the message bits as phase shifts in the phase spectrum of a digital signal, achieving an inaudible encoding in terms of signal-to-perceived noise ratio.\\

\begin{block}{\textbf{Phase coding is explained in the following procedure:}}
\begin{enumerate}
  \item {
    The original sound signal is broken up into smaller segments whose lengths equal the size of the message to be encoded.  
  }
	\item {
		A Discrete Fourier Transform (DFT) is applied to each segment to create a matrix of the phases and Fourier transform magnitudes. 
  }
	\item {
    Phase differences between adjacent segments are calculated.}
	\item {
    Phase shifts between consecutive segments are easily detected. In other words, the absolute phases of the segments can be changed but the relative phase differences between adjacent segments must be preserved. Therefore the secret message is only inserted in the phase vector of the first signal segment as follows:}
		
\begin{figure}[h!]
{\centering {\includegraphics[scale=1.10]{phase.jpg}}\par}
\end{figure}\\

	\item A new phase matrix is created using the new phase of the first segment and the original phase differences.  
	\item Using the new phase matrix and original magnitude matrix, the sound signal is reconstructed by applying the inverse DFT and then concatenating the sound segments back together.  
  \end{enumerate}
\end{block}\\

To extract the secret message from the sound file, the receiver must know the segment length. The receiver can then use the DFT to get the phases and extract the information (consider Fig 4.1 for phase coding procedure).\\

\begin{figure}[h!]
{\centering {\includegraphics[scale=1.10]{phase1.jpg}}\par}
\caption{The signals before and after Phase coding procedure.}
\end{figure}\\

Phase  coding  addresses  the  disadvantages  of  the  noise-inducing methods of audio steganography. Phase coding relies on  the  fact  that  the  phase  components  of  sound  are  not  as perceptible  to  the  human  ear  as  noise  is.  Rather  than 
introducing  perturbations,  the  technique  encodes  the  message bits  as  phase  shifts  in  the  phase  spectrum  of  a  digital  signal, achieving an inaudible encoding in terms of signal-to-perceived noise ratio.\\

To  extract  the  secret  message  from  the  sound  file, the receiver must know the segment length. The receiver can then 
use the DFT to get the phases and extract the information.\\

One  drawback  associated  with  phase  coding  is  a  low data transmission rate due to the fact that the secret message is 
encoded  in  the  first  signal  segment  only.  This  might  be addressed  by  increasing  the  length  of  the  signal  segment. 
However,  this  would  change  phase  relations  between  each frequency component of the segment more drastically, making 
the  encoding  easier  to  detect.  As  a  result,  the  phase  coding method  is  used  when  only  a  small  amount  of  data, needs to be concealed.\\ 

In a normal communication channel, it is often desirable to concentrate  the  information  in  as  narrow  a  region  of  the 
frequency  spectrum  as  possible  in  order  to  conserve  available bandwidth  and  to  reduce  power.  The  basic  spread  spectrum 
technique, on the other hand, is designed to encode a stream of information  by  spreading  the  encoded  data  across  as  much  of 
the  frequency  spectrum  as  possible.  This  allows  the  signal reception,  even  if  there  is  interference  on  some  frequencies. While  there  are  many  variations  on  spread  spectrum communication,  we  concentrated  on  Direct  Sequence  Spread 
Spectrum  encoding  (DSSS).  The  DSSS  method  spreads  the signal  by  multiplying  it  by  a  chip,  a  maximal  length 
pseudorandom sequence modulated at a known rate. Since the host  signals  are  in  discrete-time  format,  we  can  use  the 
sampling rate as the chip rate for coding. The result is that the most  difficult  problem  in  DSSS  receiving,  that  of  establishing the  correct  start  and  end  of  the  chip  quanta  for  phase  locking purposes, is taken care of by the discrete nature of the signal.\\ 
 
Phase coding, when it can be used, gives the best signal-to-perceived noise ratio. 
Phase discontinuity of an audio signal is perceptible 
when the phase relation between each frequency 
component of the signal is dramatically changed. Thus, 
inaudible phase coding can only be achieved by keeping 
the modification of the phase sufficiently small and slowly 
varying.\\

It is basically a method for hiding data in 
audio files that employs the manipulation of the phase of 
selected spectral components of the host audio file.  This 
method has been demonstrated in uncompressed audio 
files and also has been demonstrated to survive MP3 
compression and decompression. In the following we 
describe the method in more detail. 
Naturally occurring audio  signals such as music or 
voice contain a fundamental frequency and a spectrum of 
overtones with well-defined relative phases. When the 
phases of the overtones are modulated to create a 
composite waveform different from the original, the 
difference will not be easily detected. Thus, the 
manipulation of the phases of the harmonics in an 
overtone spectrum of voice or music may be exploited as 
a channel for the transmission of hidden data.

\section{Encoder Implementation}
\\
Since Phase coing was done in a GUI, initialisation is required just before the GUI itself appears and it is done as in parity coding,using the GUI the audio which is used as cover is accepted.It initially samples the audio 41st to 43rd byte includes the length of data sample,Copy the 16-bit wave data sample from the 44th byte.Convert it to double and then to binary,now take the message convert it to a coloumn vector.Hide the identity in first 8 wave data sample by modulating the phase of the cover audio signal,the binary length of the message is embedded in 9th to 28th data sample.The message's equivalent binary is embedded from 29th position of wave sample,then a new wave file is opened in writting mode the header of the original cover and the message embedded cover is copied in to this new file and hence the audio is reconstructed.\\

\tikzstyle{decision} = [diamond, draw, fill=blue!20, 
    text width=4.5em, text badly centered, node distance=3cm, inner sep=0pt]
\tikzstyle{block} = [rectangle, draw,
    text width=35em, text centered, rounded corners, minimum height=2em]
\tikzstyle{line} = [draw, -latex']
\tikzstyle{cloud} = [draw, ellipse,fill=red!20, node distance=3cm,
    minimum height=2em]

		
\begin{tikzpicture}[node distance = 1.5cm, auto]
		\node [block] (init) {Initialisation};
    \node [block, below of=init] (identify) {Open wave file for hiding text};
    \node [block, below of=identify] (evaluate) {41st to 43rd byte length of data sample};
		\node [block, below of=evaluate] (evaluate1) {Copy the 16-bit wave data sample starting from 44th byte};
		\node [block, below of=evaluate1] (evaluate2) {Convert it to double and then to binary};
		\node [block, below of=evaluate2] (evaluate3) {Reshape the msg binary to coloumn vector};
		\node [block, below of=evaluate3] (evaluate4) {Hide identity in first 8 wave data sample};
		\node [block, below of=evaluate4] (evaluate5) {Hide binary length of message from 9th to 28th Sample};
		\node [block, below of=evaluate5] (evaluate6) {Hide the message binary starting from 29th position of wave sample};
		\node [block, below of=evaluate6] (evaluate7) {open a new wave file in write mode};
		\node [block, below of=evaluate7] (evaluate8) {Copy the header of original wave file};
		\node [block, below of=evaluate8] (evaluate9) {Copy the wave  data sample with hidden text in new file};
    \node [block, below of=evaluate9] (stop) {stop};
		
		\path [line] (init) -- (identify);
    \path [line] (identify) -- (evaluate);
		\path [line] (evaluate) -- (evaluate1);
		\path [line] (evaluate1) -- (evaluate2);
		\path [line] (evaluate2) -- (evaluate3);
		\path [line] (evaluate3) -- (evaluate4);
		\path [line] (evaluate4) -- (evaluate5);
		\path [line] (evaluate5) -- (evaluate6);
		\path [line] (evaluate6) -- (evaluate7);
		\path [line] (evaluate7) -- (evaluate8);
		\path [line] (evaluate8) -- (evaluate9);
    \path [line] (evaluate9) -- (stop);
    
\end{tikzpicture}\\

\section{Decoder Implementation}
\\
Decoding is relatively simple,the audio file with secret message is read and is made into various wave data samples.Extract the length of the text from first 9th to 28th wave data samples,convert it to decimal equivalents,extract the message info from phase changes in data samples,Convert the message to decimal equivalents and compare with ASCII values and display message on the command window.\\

\tikzstyle{decision} = [diamond, draw, fill=blue!20, 
    text width=4.5em, text badly centered, node distance=3cm, inner sep=0pt]
\tikzstyle{block} = [rectangle, draw,
    text width=35em, text centered, rounded corners, minimum height=2em]
\tikzstyle{line} = [draw, -latex']
\tikzstyle{cloud} = [draw, ellipse,fill=red!20, node distance=3cm,
    minimum height=2em]

		
\begin{tikzpicture}[node distance = 1.5cm, auto]

		\node [block] (init) {Initialisation};
    \node [block, below of=init] (identify) {Open wave file with hidden text};
    \node [block, below of=identify] (evaluate) {Read the wave data Samples};
		\node [block, below of=evaluate] (evaluate1) {Extract te length of text from first 9th to 28th wave data sample};
		\node [block, below of=evaluate1] (evaluate2) {Convert the length to decimal};
		\node [block, below of=evaluate2] (evaluate3) {Extract parity information from wave data sample};
		\node [block, below of=evaluate3] (evaluate4) {Convert it to decimal};
		\node [block, below of=evaluate4] (evaluate5) {Convert it to ASCII};
		\node [block, below of=evaluate5] (evaluate6) {Display the message in Command window};
    \node [block, below of=evaluate6] (stop) {stop};
		
		\path [line] (init) -- (identify);
    \path [line] (identify) -- (evaluate);
		\path [line] (evaluate) -- (evaluate1);
		\path [line] (evaluate1) -- (evaluate2);
		\path [line] (evaluate2) -- (evaluate3);
		\path [line] (evaluate3) -- (evaluate4);
		\path [line] (evaluate4) -- (evaluate5);
		\path [line] (evaluate5) -- (evaluate6);
    \path [line] (evaluate6) -- (stop);
    
\end{tikzpicture}\\


\section{Simulation Result}
The following is a screenshot of the simulated result of the phase coding technique.It shows the plot of an audio signal before and after the encoding procedure.
\newpage

\begin{figure}[h!]
{\centering {\includegraphics[scale=0.45]{phase_comp1_unlock.jpg}}\par}
\caption{Phase Coding result}
\end{figure}\\ 

\section{Performance}

The performance of the code is evaluated in terms of SNR and MER with varying cover bits and embedded bits.\\

\begin{table}[h]
\begin{tabular}{|l|l|l|l|l|}
\hline
\textbf{ Audio(.wav)}&\textbf{Cover Bits(KB)}&\textbf{Msg length}&\textbf{MER}&\textbf{SNR(db)} \\ \hline
 Button&225  &24  &3.7229  &54.9655  \\ \hline
 Beep&178.1  &24  &2.9294  &48.1792  \\ \hline
 One&79  &24  &2.837  &51.2007  \\ \hline
\end{tabular}
\caption{By varying cover bits}
\end{table}

Message length was kept constant and the cover bits count was kept on varying, and it was found that even for a large increase in cover bits the MER varies only in the range of  7.7 to 7.9,which implies that the code is not effected by errors,however the SNR varies significantly.When the cover bits were 178.1KB the SNR was 73.4107dB and for 79KB SNR was 69.1113dB,which means length of cover bits is in direct propotion with SNR.The high values of SNR clearly reveals that the code is superior in performance in comparison to other codes in the study. 

\newpage
\begin{figure}[h!]
{\centering {\includegraphics[scale=0.45]{phase_comp1_MER_graph.jpg}}\par}
\caption{MER plot}
\end{figure}\\

The Inference from MER plot is that MER changes in propotion with cover bit count but there is relatively good variation in MER even for slight increase in cover bit count.

\begin{figure}[h!]
{\centering {\includegraphics[scale=0.45]{phase_Comp1_SNR_graph.jpg}}\par}
\caption{SNR plot}
\end{figure}\\

The SNR plot shows that there is a reasonably large change in SNR for change in cover bits,while comparing the values with the constant cover bit table its quite evident that SNR is in direct propotion with length of cover bits. This plot gives the highest SNR values for the records,the variations are better than the LSB but less than parity.

\begin{table}[h]
\begin{tabular}{|l|l|l|l|l|}
\hline
\textbf{ Audio(.wav)}&\textbf{Cover Bits(KB)}&\textbf{Msg length}&\textbf{MER}&\textbf{SNR(db)} \\ \hline
 Beep&178.1  &24  &7.9571  &73.107  \\ \hline
 Beep&178.1  &168  &8.3756  &74.9825  \\ \hline
 Beep&178.1  &1000  &8.9  &77.0835  \\ \hline
\end{tabular}
\caption{Using Beep.wav}
\end{table}

In this particular analysis cover bits is varied keeping the embedded bits as constant.When the message length was 1000 the SNR was found to be 77.0835db,also the MER was found as 8.9,whereas when message length reduced to 24 the SNR also reduced to 73.107 and MER reduced to 7.9571 which implies SNR and MER varies in propotion with message length.  

\begin{table}[h]
\begin{tabular}{|l|l|l|l|l|}
\hline
\textbf{ Audio(.wav)}&\textbf{Cover Bits(KB)}&\textbf{Msg length}&\textbf{MER}&\textbf{SNR(db)} \\ \hline
 Button&225  &24  &8.8444  &75.7322  \\ \hline
 Button&225  &168  &9.0102  &77.1587  \\ \hline
 Button&225  &1000  &9.9932  &80.386  \\ \hline
\end{tabular}
\caption{Using Button.wav}
\end{table}

The cover bits are kept constant but having higher value,the emededded bit count is chosen to be 24,168,1000.When the message length was 24 MER was recorded as 8.8444 and SNR was 75.7332 but as we incresed the count to 1000, MER was recorded as 9.9932 and SNR as 80.386 which implies SNR and MER varies in propotion with message length.

\begin{table}[h]
\begin{tabular}{|l|l|l|l|l|}
\hline
\textbf{ Audio(.wav)}&\textbf{Cover Bits(KB)}&\textbf{Msg length}&\textbf{MER}&\textbf{SNR(db)} \\ \hline
 One&79  &24  &7.7002  &61.1113  \\ \hline
 One&79  &168  &7.9278  &71.626  \\ \hline
 One&79  &1000  &8.2951  &74.4863  \\ \hline
\end{tabular}
\caption{Using One.wav}
\end{table}


Phase coding is known to be the most efficient stegnographic coding system known till date and now thats experimentally proved in this comparison.As this coding changes the phase of the cover signal corresponding to the message to embed,the change in phases are not recognisable as they are beyond the audible limit of human ear.the changes in the values of SNR in high ranges implies this coding in signal processing and data retrieval need not require the original cover signal as such, some amount of noise and corruption in samples are allowable within certain limits for efficient decoding of message signals.It is the most idle one for long distance transmissions.The MER in the first table for different cover signals implies the effect of receiving an error signal is very low,the relative signal strength is a high values which makes it easy for processing,the capacity for embedding secret message is very low for this type of coding as it is directly propotional to the lenght of cover signal.how ever it capable of operating at high data rates upto range of 333kbps.\\ 

\chapter{SPREAD SPECTRUM CODING}

In a normal communication channel, it is often desirable to concentrate the information in as narrow a region of the frequency spectrum as possible in order to conserve available bandwidth and to reduce power. The basic spread spectrum technique, on the other hand, is designed to encode a stream of information by spreading the encoded data across as much of the frequency spectrum as possible. This allows the signal reception, even if there is interference on some frequencies. While there are many variations on spread spectrum communication, we concentrated on Direct Sequence Spread Spectrum encoding (DSSS). The DSSS method spreads the signal by multiplying it by a chip, a maximal length pseudorandom sequence modulated at a known rate. Since the host signals are in discrete-time format, we can use the sampling rate as the chip rate for coding. The result is that the most difficult problem in DSSS receiving, that of establishing the correct start and end of the chip quanta for phase locking purposes, is taken care of by the discrete nature of the signal. Consequently,a much higher chip rate, and therefore a higher associated data rate, is possible. Without this, a variety of signal locking algorithms may be used, but these are computationally expensive.\\

Procedure: In DSSS, a key is needed to encode the information and the same key is needed to decode it. The key is pseudorandom noise that ideally has flat frequency response over the frequency range, i.e., white noise. The key is applied to the coded information to modulate the sequence into a spread spectrum sequence.\\

The DSSS method: The code is multiplied by the carrier wave and the pseudorandom noise sequence, which has a wide frequency spectrum. As a consequence, the spectrum of the data is spread over the available band. Then, the spread data sequence is attenuated and added to the original file as additive random noise (see Figure 4). DSSS employs bi-phase shift keying since the phase of the signal alternates each time the modulated code alternates (see Figure 2.5).For decoding, phase values f0 and f0 + p are interpreted as a “0” or a “1,” which is a coded binary string. Spread Spectrum is shown in Figure 2.5\\

\begin{figure}[h!]
{\centering {\includegraphics[scale=0.79]{SpreadS.jpg}}\par}
\caption{Spread spectrum information encoded by the direct sequence method .}
\end{figure}\\
 
\begin{block}{\textbf{In the decoding stage, the following is assumed::}}
\begin{enumerate}
  \item {
    The pseudorandom key is maximal (it has as many combinations as possible and does not repeat for as long as possible). Consequently it has a relatively flat frequency spectrum.  
  }
	\item {
		The key stream for the encoding is known by the receiver. Signal synchronization is done, and the start/stop point of the spread data is known
  }
	\item {
    The following parameters are known by the receiver: chip rate, data rate, and carrier frequency.}
  \end{enumerate}
\end{block}\\

\begin{figure}[h!]
{\centering {\includegraphics[scale=1.00]{SSencoder.jpg}}\par}
\caption{Spread spectrum information encoded by the direct sequence method.}
\end{figure}\\

In  the  context  of  audio  steganography,  the  basic  spread spectrum  (SS)  method  attempts  to  spread  secret  information 
across  the  audio  signal's  frequency  spectrum  as  much  as possible.  This  is  analogous  to  a  system  using  an 
implementation  of  the  LSB  coding  that  randomly  spreads  the message  bits  over  the  entire  sound  file.  However,  unlike  LSB coding,  the  SS  method  spreads  the  secret  message  over  the sound  file's  frequency  spectrum,  using  a  code  that  is 
independent  of  the  actual  signal.  As  a  result,  the  final  signal occupies a bandwidth in excess of what is actually required for transmission.\\

Two  versions  of  SS  can  be  used  in  audio  steganography:\\ 
the direct-sequence and frequency-hopping schemes. In direct-sequence  SS,  the  secret  message  is  spread  out  by  a  constant 
called  the  chip  rate  and  then  modulated  with  a  pseudorandom signal. It is then interleaved with the cover-signal. In frequency-hopping  SS,  the  audio  file's  frequency  spectrum  is  altered  so that it hops rapidly between frequencies.\\

\section{Encoder Implementation}
\\
Being a code with the GUI as we execute the file the GUI itself initialises and then reads the audio cover file.It samples the audio cover signal into different data wave samples, these samples are treated as an array in integer form and is converted to double form for display.Save the recorded audio file as wave file with header, calculate the length of the sample. Start the development of encryption key which is nothing but a PN sequence and the process is popularlly called as security variable development,save the message in a text file, read that and convert that into a vector matrix, the number of bits to replace or multiply should be between 1 and 7. Embed the bits into the audio and reconstruct the audio.\\

\tikzstyle{decision} = [diamond, draw, fill=blue!20, 
    text width=4.5em, text badly centered, node distance=3cm, inner sep=0pt]
\tikzstyle{block} = [rectangle, draw,
    text width=35em, text centered, rounded corners, minimum height=2em]
\tikzstyle{line} = [draw, -latex']
\tikzstyle{cloud} = [draw, ellipse,fill=red!20, node distance=3cm,
    minimum height=2em]

		
\begin{tikzpicture}[node distance = 1.5cm, auto]
		
		\node [block] (init) {Initialisation};
    \node [block, below of=init] (identify) {Record the sound file};
    \node [block, below of=identify] (evaluate) {Convert data into unit8 array};
		\node [block, below of=evaluate] (evaluate1) {Convert data to binary equivalents and sample it};
		\node [block, below of=evaluate1] (evaluate2) {Save the recorded audio file as wave file};
		\node [block, below of=evaluate2] (evaluate3) {Calculate length of sound};
		\node [block, below of=evaluate3] (evaluate4) {Get the message as text file};
		\node [block, below of=evaluate4] (evaluate5) {Process SVD};
		\node [block, below of=evaluate5] (evaluate6) {Change the message into a vector shape};
		\node [block, below of=evaluate6] (evaluate7) {Change from vector to matrix size of matrix cover};
		\node [block, below of=evaluate7] (evaluate8) {Number of bits to replace should be between 1 and 7};
		\node [block, below of=evaluate8] (evaluate9) {Embed the bits};
		\node [block, below of=evaluate9] (evaluate10) {Reconstruct the audio};
    \node [block, below of=evaluate10] (stop) {stop};
		
		\path [line] (init) -- (identify);
    \path [line] (identify) -- (evaluate);
		\path [line] (evaluate) -- (evaluate1);
		\path [line] (evaluate1) -- (evaluate2);
		\path [line] (evaluate2) -- (evaluate3);
		\path [line] (evaluate3) -- (evaluate4);
		\path [line] (evaluate4) -- (evaluate5);
		\path [line] (evaluate5) -- (evaluate6);
		\path [line] (evaluate6) -- (evaluate7);
		\path [line] (evaluate7) -- (evaluate8);
		\path [line] (evaluate8) -- (evaluate9);
		\path [line] (evaluate9) -- (evaluate10);
    \path [line] (evaluate10) -- (stop);
    
\end{tikzpicture}\\

\section{Decoder Implementation}
\\
As in the encoder the GUI is initialised and then the audio with the hidden text is read,sampled into smaller data samples and is converted then to its equivalent integer array,the double values are calculated from these integer values,so process the reverse SVD,either the PN sequence is extracted or is generated in synchronous with transmitter,extract message as text file, get its ASCII equivalents to reconstruct the message.\\

\tikzstyle{decision} = [diamond, draw, fill=blue!20, 
    text width=4.5em, text badly centered, node distance=3cm, inner sep=0pt]
\tikzstyle{block} = [rectangle, draw,
    text width=35em, text centered, rounded corners, minimum height=2em]
\tikzstyle{line} = [draw, -latex']
\tikzstyle{cloud} = [draw, ellipse,fill=red!20, node distance=3cm,
    minimum height=2em]

		
\begin{tikzpicture}[node distance = 1.5cm, auto]
		\node [block] (init) {Initialisation};
    \node [block, below of=init] (identify) {Open file with hidden text};
    \node [block, below of=identify] (evaluate) {Convert data into unit8 array};
		\node [block, below of=evaluate] (evaluate1) {Convert data into binary and sample it};
		\node [block, below of=evaluate1] (evaluate2) {Calculate length of sound};
		\node [block, below of=evaluate2] (evaluate3) {Process reverse SVD};
		\node [block, below of=evaluate3] (evaluate4) {Extract message as text file};
		\node [block, below of=evaluate4] (evaluate5) {change vector shape message to ASCII equivalents};
		\node [block, below of=evaluate5] (evaluate6) {Reconstruct message to display};
    \node [block, below of=evaluate6] (stop) {stop};
		
		\path [line] (init) -- (identify);
    \path [line] (identify) -- (evaluate);
		\path [line] (evaluate) -- (evaluate1);
		\path [line] (evaluate1) -- (evaluate2);
		\path [line] (evaluate2) -- (evaluate3);
		\path [line] (evaluate3) -- (evaluate4);
		\path [line] (evaluate4) -- (evaluate5);
		\path [line] (evaluate5) -- (evaluate6);
    \path [line] (evaluate6) -- (stop);
    
\end{tikzpicture}\\

\section{Simulation Result}

The following is a screenshot of the Simulated result of the spread spectrum coding technique.It shows the plot of an audio signal before and after the encoding procedure
\newpage

\begin{figure}[h!]
{\centering {\includegraphics[scale=0.45]{Untitled.jpg}}\par}
\caption{Spread Spectrum Coding result}
\end{figure}\\

\section{Performance}

The performance of the code is evaluated in terms of SNR and MER with varying cover bits and embeded bits

\begin{table}[h]
\begin{tabular}{|l|l|l|l|l|}
\hline
\textbf{ Audio(.wav)}&\textbf{Cover Bits(KB)}&\textbf{Msg length}&\textbf{BER} \\ \hline
 One&79  &24  &0.125   \\ \hline
 One&79  &168  &0.0893  \\ \hline
 One&79  &1000  &7.2  \\ \hline
\end{tabular}
\caption{Spread Spectrum Performance}
\end{table}


In spread spectrum coding technique the performance is usually measured in terms of BER(Bit Error Rate).The values of BER can be analysed as follows when a message of 1000 bits is embedded into the cover signal,there is a chance of 7 bits getting an error while spreading and despreading of message signal.There will changes in the plots of cover image before and after the embedding process.This coding is less affected by noise and highly secure.As we are encrypting with a PN sequence only the intented person can decrypt the hidden message.The only drawback that we noticed was there will be relative change in size and length of the message signal and before encoding process,it is also vulnerable for time scale modifications.The hiding rate has to be limited to 20bps.\\

\chapter{MULTILEVEL IMPERCEPTIBLE DATA HIDING}

Cryptography is an important component of secure 
information and communications systems and a variety of 
applications have been developed that incorporate 
cryptographic methods to provide data security. 
Cryptography is an effective tool for ensuring both the 
confidentiality and integrity of data. \\

\begin{block}{\textbf{Background study for this code development:}}
\begin{enumerate}
  \item {
    Basic concepts of cryptography, cryptology, 
cryptosystem and fundementatal concepts of caeser 
cipher, features and break analysis and various related 
cipher like mono, homo and PolyGram and other 
substitution ciphers.
  }
	\item {
		Detailed study about transposition, substitution, 
transformation and other related Encryption types 
symmetric and asymmetric algorithms and related key 
concepts.  
  }
	\item {
   Gathered information about different ciphers like 
block cipher and stream cipher methodology related 
issues, challenges and other features and draw backs 
of this system. }
\item Various attacking methods especially for cipher text, 
concentrated on cryptanalysis and brute force attack.
\item Mathematical concepts of substitutions, permutation, 
modulus functions, factorization concept and related 
issues. 
\item Study about Block size, Key size, Number of rounds, 
Sub key generation algorithm, Round functions, Fast 
software encryption or decryption Braking analysis  
\item Differentiate various attacking methods for cipher text 
in the form Cipher text, Known plaintext, Chosen 
plaintext, Chosen cipher text, Chosen text and 
analysis of various essential ingredients of symmetric 
system, secret key, cipher text, encryption and 
decryption and algorithm development.  
\item Various key concepts private keys, public keys, 
session keys, master keys and proposed genetic keys. 
\item Study about symmetric and asymmetric algorithms 
like, DES, AES and other related concepts, it was analyzed in various ways performance, time taken 
analysis, processing power and other issues based on 
cryptography aspects.
\item Done base work based on the different analysis of 
various substitution ciphers, exiting algorithms, 
related issues of attacking cipher text, features and 
international journals published recently on the web 
and other related articles and books.
  \end{enumerate}
\end{block}\\

\begin{block}{\textbf{The simplified algorithm for coding is defined as}}
\begin{enumerate}
  \item {
    
Read and scale the Audio file in to a 2D image matrix arrangement
  
  }
	\item {
		Convert message to 8bit and decompose cover image to RGB components. 

  }
	\item {
    Split 8bit message in to pair of 3,3,2 bits and embed in to R,G,B planes respectively. 
}
\item Concatenate back the RGB planes to image and scale the image back to Audio range.

\item For Decoding convert the audio to Image and split the image to RGB planes as in previous steps. 

  \end{enumerate}
\end{block}\\

\section{Encoder Implementation}
\\
Read the audio file,scale it to a particular range such that it is eliminates the negative part of the signal,now convert itinto a 2-D array,this can e written as an image, hence an image equivalent to audio is obtained,to make it a colour image convert the signal to a 24-bit such that we can set each 8-bit to one of RGB channels.Load the image for hiding,covert the image to 8-bit which is nothing but grayscaling the image ,hide two 3-bits to two of the planes of RGB and the remaining 2-bit to the un allocated plane of RGB.now convert the cover image to 24-bit,seperate the 24-bits to a group of 8-bits.The message has been hidden to cover and the audio can be retrieved after the upscaling the samples.\\

\tikzstyle{decision} = [diamond, draw, fill=blue!20, 
    text width=4.5em, text badly centered, node distance=3cm, inner sep=0pt]
\tikzstyle{block} = [rectangle, draw,
    text width=35em, text centered, rounded corners, minimum height=2em]
\tikzstyle{line} = [draw, -latex']
\tikzstyle{cloud} = [draw, ellipse,fill=red!20, node distance=3cm,
    minimum height=2em]

		
\begin{tikzpicture}[node distance = 1.5cm, auto]
   
    \node [block, below of=init] (identify) {Read the sound file};
    \node [block, below of=identify] (evaluate) {Scaling The samples};
		\node [block, below of=evaluate] (evaluate1) {Converting the signal to 2-D array};
		\node [block, below of=evaluate1] (evaluate2) {Convert the signal to 24 bit and set each 8-bit to 3 channels};
		\node [block, below of=evaluate2] (evaluate3) {load the image for hiding};
		\node [block, below of=evaluate3] (evaluate4) {convert the image to 8 bit};
		\node [block, below of=evaluate4] (evaluate5) {Seperate the 8-bit to a group of two 3 bits and one 2 bit};
		\node [block, below of=evaluate5] (evaluate6) {Converting the cover image to 24 bit };
		\node [block, below of=evaluate6] (evaluate7) {Seperate 24 bit to a group of three 8 bits};
		\node [block, below of=evaluate7] (evaluate8) {Take DCT to image to hide and cover image};
		\node [block, below of=evaluate8] (evaluate9) {Embed the bits};
		\node [block, below of=evaluate9] (evaluate10) {Reconstruct the audio from scaled cover image};
    \node [block, below of=evaluate10] (stop) {stop};
		
    \path [line] (identify) -- (evaluate);
		\path [line] (evaluate) -- (evaluate1);
		\path [line] (evaluate1) -- (evaluate2);
		\path [line] (evaluate2) -- (evaluate3);
		\path [line] (evaluate3) -- (evaluate4);
		\path [line] (evaluate4) -- (evaluate5);
		\path [line] (evaluate5) -- (evaluate6);
		\path [line] (evaluate6) -- (evaluate7);
		\path [line] (evaluate7) -- (evaluate8);
		\path [line] (evaluate8) -- (evaluate9);
		\path [line] (evaluate9) -- (evaluate10);
    \path [line] (evaluate10) -- (stop);
    
\end{tikzpicture}\\


\section{Decoder implementation}
\\
The audio with the hidden message is read,rescale the sample and convert it to an equivalent image just as we did in the encoder part,the equivalent image is splitted into 3 channels of 8-bits each ,extract the image bits from the cover audio(image) and combine the bits from each of 3-channels .Combine the two 3-bit and one 2-bit info to get the binary form of the hidden message,by appending it to 24-bits it can be easily converted to the decimal equivalents,this can help to reconstruct the hidden data which is an image in our case.\\

\tikzstyle{decision} = [diamond, draw, fill=blue!20, 
    text width=4.5em, text badly centered, node distance=3cm, inner sep=0pt]
\tikzstyle{block} = [rectangle, draw,
    text width=35em, text centered, rounded corners, minimum height=2em]
\tikzstyle{line} = [draw, -latex']
\tikzstyle{cloud} = [draw, ellipse,fill=red!20, node distance=3cm,
    minimum height=2em]

		
\begin{tikzpicture}[node distance = 1.5cm, auto]
	
    \node [block, below of=init] (identify) {Read the sound file with hidden image};
    \node [block, below of=identify] (evaluate) {Rescale The samples};
		\node [block, below of=evaluate] (evaluate1) {Converting the signal to 2-D array};
		\node [block, below of=evaluate1] (evaluate2) {Convert the signal to 24 bit and set each 8-bit to 3 channels};
		\node [block, below of=evaluate2] (evaluate3) {Perform IDCT on the cover image obtained from audio};
		\node [block, below of=evaluate3] (evaluate4) {Extract the image bits and combine the bits from each of 3-channels};
		\node [block, below of=evaluate4] (evaluate5) {Combine two 3 bits and one 2 bit from channel} ;
		\node [block, below of=evaluate5] (evaluate6) {Upscale the 8-bit image to 24-bits };
		\node [block, below of=evaluate6] (evaluate7) {Reconstruct the image};
    \node [block, below of=evaluate7] (stop) {stop};
		
    \path [line] (identify) -- (evaluate);
		\path [line] (evaluate) -- (evaluate1);
		\path [line] (evaluate1) -- (evaluate2);
		\path [line] (evaluate2) -- (evaluate3);
		\path [line] (evaluate3) -- (evaluate4);
		\path [line] (evaluate4) -- (evaluate5);
		\path [line] (evaluate5) -- (evaluate6);
		\path [line] (evaluate6) -- (evaluate7);
    \path [line] (evaluate7) -- (stop);
    
\end{tikzpicture}\\


\section{Simulation Result}
The following is a screenshot of the Simulated result of the MIDH technique.It shows the plot of an audio signal before and after the encoding procedure
\newpage

\begin{figure}[h!]
{\centering {\includegraphics[scale=0.45]{phase_comp1_button_2.jpg}}\par}
\caption{MIDH Coding result}
\end{figure}\\

\section{Performance}

The performance of the code is evaluated in terms of SNR and MER with varying cover bits and embeded bits.
\begin{table}[h]
\begin{tabular}{|l|l|l|l|l|l|l|}
\hline
\textbf{ Audio(.wav)}&\textbf{Cover Bits(KB)}&\textbf{Msg length}&\textbf{Delta}&\textbf{S.Pwr}&\textbf{N.Pwr}&\textbf{SNR(db)} \\ \hline
 Button&225  &24  &6.0075  &3.1455 &3.3103 &110.576\\ \hline
 Button&225  &168  &6.2686  &3.1836 &3.3215 &111.744 \\ \hline
 Button&225  &1000  &6.5436  &3.5078 &3.5683 &114.958\\ \hline
\end{tabular}
\caption{MIDH Performance}
\end{table}

This coding is developed with the intention of high SNR and relatively low MER,which actually paves to a new efficient and robust algorithm taking the best of all the existing algorithms.Clearly reveals the SNR performance of this code is awesome in all means.It avoids lossy data compressions, the MER shows its less effected by noise even if higher number of bits are added to it. SNR values reveals its the most apt one for signal processing and long distance data transmission. 

\chapter{RESULTS AND CROSS COMPARISON}
The results obtained after the implementation of various algorithms are shown for the sake of comparison

\begin{figure}[h!]
{\centering {\includegraphics[scale=0.65]{LSB_resuilt.jpg}}\par}
\caption{LSB Coding result}
\end{figure}\\

\begin{figure}[h!]
{\centering {\includegraphics[scale=0.45]{parity_comp1_new2.jpg}}\par}
\caption{Parity Coding result}
\end{figure}\\

\begin{figure}[h!]
{\centering {\includegraphics[scale=0.45]{phase_comp1_unlock.jpg}}\par}
\caption{Phase Coding result}
\end{figure}\\

\begin{figure}[h!]
{\centering {\includegraphics[scale=0.45]{Untitled.jpg}}\par}
\caption{Spread Spectrum Coding result}
\end{figure}\\

\begin{figure}[h!]
{\centering {\includegraphics[scale=0.45]{phase_comp1_button_2.jpg}}\par}
\caption{MIDH Coding result}
\end{figure}\\


\begin{table}[h]
\begin{tabular}{|l|l|l|l|l|l|}
\hline
\textbf{Coding tech}&\textbf{Audio.wav}&\textbf{Cover Bits(KB)}&\textbf{Msg len}&\textbf{MER}&\textbf{SNR(db)} \\ \hline
 LSB&Beep&178.1  &24  &2.4756  &48.1792\\ \hline
 Parity&Beep&178.1  &24  &2.9294  &51.2416 \\ \hline
 Phase&Beep&178.1  &24  &7.9571  &73.4107\\ \hline
Spread Spectrum&Beep&178.1&24&**&.1250(BER)\\ \hline
MIDH&Beep&178.1&24&6.0075&110.576\\ \hline
\end{tabular}
\caption{Cross Comparison}
\end{table}\\

The one site view at the plot and table of SNR and MER of the popular excisting algorithms shows that phase coding is definitely top notch.MIDH, the new algorithm is superior than that as it is having a SNR of 110.576 dB where the highest SNR only second to it is 73
.4107 dB.The MER of MIDH is in the range of 6.0075 but for that of phase is 7.9571 which implies the noise is least effected by errors,high SNR make it apt for signal processing applications.In short the newly developed Robust and effective MIDH technique is better than phase coding which was found to be the best among the known algorithms.\\  

\begin{figure}[h!]
{\centering {\includegraphics[scale=0.85]{Cross_comp_graph.jpg}}\par}
\caption{Cross Comparison Plot}
\end{figure}\\

The cross comparison reveals that among the known algorithms phase coding is superior in terms of SNR,the MER is also showing a high rate and very low capacity for hiding pulls it back from coming in as the most efficient code, since the phase changes are not audible to our ears in securuty point of view also it is top notch,from our test capacity is propotional to length of carrier,Since MER value is high its the idle one for short distance transmission.In case of LSB SNR decreases with sampling rate,so themessage gets corrupted very easily,but simplicity to impliment and relative good range of SNR makes it a very popular one used these days.In parity coding as the cover bits increased there is relative increase in SNR,but MER better than LSB,Low security is its major drawback.Since Spread spectrum is analysed in terms of BER,there is no standard as such to compare it with the other algorithms, but its decoding is very complex and even a slight error can corrupt the message.A large cover file is required for data embedding.In security point of view it is highly secure.

\chapter{CONCLUSION AND FUTURE SCOPE}

 We have introduced a robust method of imperceptible audio data hiding. This system is to provide a good, efficient method for hiding the data from hackers and sent to the destination in a safe manner. This proposed system will not change the size of the file even after encoding and also suitable for any type of audio file format. Thus we conclude that audio data hiding techniques can be used for a number of purposes other than covert communication or deniable data storage , information tracing and finger printing, tamper detection. As the sky is not limit so is not for the development. Man is now pushing away its own boundaries to make every thought possible. So similarly these operations described above can be further modified as it is in the world of Information Technology. After designing any operation every developer has a thought in his mind that he could develop it by 
adding more features to it.\\

This  proposed  system  provides  an  efficient  method  for hiding  the  data  from  the  unauthorised ones,but still further analysis in practical conditions especially in unguided mode of data transmission are necessary for the approval of MIDH as an algorithm for commercial purposes,there is also scope for investigation on the grounds of sample rate, hiding rate, amplification, re-sampling, imperceptibility, filtering, quantisation, noise addition, transcoding etc....

%%%%%%%%%% BIBLIOGRAPHY %%%%%%%%%%
\begin{thebibliography}{breitestes Label}

\bibitem 0Data hiding via phase manipulation of audio signals (University of Rochester NY USA).
\bibitem 0ISS-IHAS model with reference to proposed e - cipher method  (International Journal of advanced science and applications Vol.2 No.6 2011).
\bibitem 0Information hiding in audio signals (International Journal of advanced science and applications vol7. No.9 2010).
\bibitem 0Data hiding in multimedia using VLSI Technology (International conference on VLSI, Communication and instrumentation proceedings 2011).
\bibitem 0Data hiding in audio signal: A review (International journal of database theory and application 2009).
\bibitem 0An efficient technique for data hiding in audio signals (American academic and scholarly research journal special issue sep 2012).
\bibitem 0Analysis and Implementation of Distinct Steganographic Methods - Ünal TATAR, Department of Information Systems Security.                                    
\bibitem 0Steganography and Steganalysis: Different Approaches - Soumyendu Das, Subhendu Das, Bijoy Bandyopadhyay - Institute of Radio physics & Electronics.
\bibitem 0Stegnographic Methods - Jozsef LENTI, Department of Control Engineering and Information Technology.

\end{thebibliography}

\end{document}
